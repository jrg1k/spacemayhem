\documentclass[11pt]{article}

\usepackage{graphicx}
\usepackage[numbers]{natbib}

\title{Assignment 3 report - INF-1400}
\author{Jørgen Kristensen \and Ivan Moen}


\begin{document}

    \maketitle


    \section{Introduction}

    In this assignment we were to make a \emph{Space mayhem} clone using OOP principles, with a specific requirement being inheriting from the Sprite class. This assignment also emphasizes good coding practices with well documented classes and methods.


    \section{Background}


    For this assignment, Python 3.9.2 was used. For server-to-client communication, we used the library named \emph{Asyncio}. For client frontend we are using the library Pygame to draw the game object on the screen of the player.


    \section{Design}

    \emph{Give an overarching view of the structure of your solution.}

Describe how your objects fit together, a figure like figure \ref{umlfig} must be included, and you should refer to it in the design section.

%    \begin{figure}[h]
%        \centering
%        \includegraphics[width=\textwidth]{UML.png}
%        \caption{Example UML diagram, taken from \cite{umlsource}}
%        \label{umlfig}
%    \end{figure}


    \section{Implementation}

    \emph{Describe implementation details, particularly those that are not obvious choices}

For the implementation of the paddle, the visual representation on the screen is different from the internal representation used for collision detection, by representing the paddle in this way we acheive...


    \section{Evaluation}

    \emph{Examine if your submission fulfils the requirements and what shortcomings exist.}

    In this solution, all requirements are fulfilled, but collision detection between the ball and paddle is inaccurate, due to differences between the visual representation and the implementation...


    \section{Discussion}

    \emph{Discuss what could be done better, problems you had, experiences etc. (we also appreciate feedback on the assignment or group sessions) }

    The implementation of the paddle-ball collision could be done \emph{some other way}, but due to \emph{some reason}, the current implemetation is better.

    After spending two days trying to write the report in \LaTeX, I gave up, and wrote it in Word instead.


    \section{Conclusion}

    \emph{Sum up the previous sections.}

    I have implemented a solution that fulfills the requirements, the implementation is moderately buggy, but does not crash too much.


    \bibliographystyle{plainnat}
    \bibliography{example.bib}

\end{document}
