\documentclass[11pt]{article}

\usepackage{graphicx}
\usepackage[numbers]{natbib}

\title{Assignment 3 report - INF-1400}
\author{Jørgen Kristensen \and Ivan Moen}


\begin{document}

    \maketitle


    \section{Introduction}

    In this assignment we were to make a \emph{Space mayhem} clone using OOP principles, with a specific requirement being inheriting from the Sprite class. This assignment also emphasizes good coding practices with well documented classes and methods.


    \section{Background}


    For this assignment, Python 3.9.2 was used. For server-to-client communication, we used the library named \emph{Asyncio}. For client frontend we are using the library Pygame to draw the game object on the screen of the player.


    \section{Design}

    \emph{Give an overarching view of the structure of your solution.}

    For this assignment we have one major class hierarchy of gameobjects, with a
    general layout of a gameobject (ie. a ship, planet etc.) that has two
    subclasses: One subclass is based on how the object is behaving on the server
    side, while the other is based on local representation of the object on the
    class, which means the local class naturally will inherit from the
    \emph{Sprite} class in pygame as well.

    We also used a object-oriented approach for server and client handling, thus
    one can create a new server by calling it as an object, and add new players as
    objects as well.

    Describe how your objects fit together, a figure like figure \ref{umlfig} must be included, and you should refer to it in the design section.

%    \begin{figure}[h]
%        \centering
%        \includegraphics[width=\textwidth]{UML.png}
%        \caption{Example UML diagram, taken from \cite{umlsource}}
%        \label{umlfig}
%    \end{figure}


    \section{Implementation}

    \emph{Describe implementation details, particularly those that are not obvious choices}

    Since the game is played online, most of the information is processed
    serverside. The players send certain information to the server:
    \begin{itemize}
        \item player id
        \item initial position, velocity and direction
        \item player input(combination of keyboard buttons left, right, up and
        spacebar)
    \end{itemize}

    To illustrate this client-server relation, we can see how the player input
    handling is working: the player clicks a button(lets say up arrow), the client
    will do an OR comparison with the default 0: $0000 \lor 0100 = 0100$.
    The server will then confirm this using an AND operation: $0100 \land
    0100 = 0100$
    and do the following on the remote object: increase velocity vector and reduce
    fuel.
    This new information will then be sent back to the client which will
    update the local object with the new information, and draw the new changes on
    the screen for the player to see.
    The information sent to and from server-client is coded and decoded into
    JSON using the python JSON module.


    For the implementation of the paddle, the visual representation on the screen is different from the internal representation used for collision detection, by representing the paddle in this way we acheive...


    \section{Evaluation}

    \emph{Examine if your submission fulfils the requirements and what shortcomings exist.}

    In this solution, all requirements are fulfilled, but collision detection between the ball and paddle is inaccurate, due to differences between the visual representation and the implementation...


    \section{Discussion}

    \emph{Discuss what could be done better, problems you had, experiences etc. (we also appreciate feedback on the assignment or group sessions) }

    The implementation of the paddle-ball collision could be done \emph{some other way}, but due to \emph{some reason}, the current implemetation is better.

    After spending two days trying to write the report in \LaTeX, I gave up, and wrote it in Word instead.


    \section{Conclusion}

    \emph{Sum up the previous sections.}

    I have implemented a solution that fulfills the requirements, the implementation is moderately buggy, but does not crash too much.


    \bibliographystyle{plainnat}
    \bibliography{example.bib}

\end{document}
